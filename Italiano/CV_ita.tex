\documentclass[a4paper,10pt]{article}
\usepackage[margin=1.5cm]{geometry}
\usepackage{paracol}
\usepackage{hyperref}
\usepackage{enumitem}
\usepackage{xcolor}
\usepackage{lmodern}
\usepackage{fancyhdr}

% ----------------------------
% Header/Footer
% ----------------------------
\pagestyle{fancy}
\fancyhf{}
\fancyfoot[R]{\thepage}

% ----------------------------
% Custom formatting
% ----------------------------
\newcommand{\sectionsep}{\vspace{0.5em}\hrule\vspace{0.5em}}
\newcommand{\descript}[1]{\textit{#1}}
\newcommand{\location}[1]{\textcolor{gray}{#1}}
\newcommand{\runsubsection}[1]{\textbf{#1}}

\newenvironment{tightemize}{
\begin{itemize}[leftmargin=*, topsep=0pt, itemsep=2pt, parsep=0pt]
}{\end{itemize}}

\setlength{\parskip}{0pt}

\begin{document}

% ----------------------------
% Name and contact
% ----------------------------
\begin{center}
    {\Huge \textbf{Emanuele Aquilia}}\\[1mm]
    \href{mailto:ema.aqui.02@gmail.com}{ema.aqui.02@gmail.com} \\
    \href{https://github.com/Emanuele-Aquilia}{GitHub} \textbullet{} 
    \href{https://www.linkedin.com/in/emanuele-aquilia-775072266/}{LinkedIn} \\
    {\small Ultimo aggiornamento: \today}
\end{center}
\sectionsep

% ========================
% Pagina 1
% ========================
\begin{paracol}{2}
\setlength{\columnsep}{1cm}
\columnratio{0.6,0.4}

% ----------------------------
% COLONNA SINISTRA
% ----------------------------

%%%%%%%%%%%%%%%%%%%%%%%%%%%%%%%%%%%%%%
% ESPERIENZA
%%%%%%%%%%%%%%%%%%%%%%%%%%%%%%%%%%%%%%
\section*{Esperienze}

\runsubsection{\href{https://www.policumbent.it/en/}{Policumbent}} 
\descript{| Responsabile Divisione IT} \\
\location{Set 2025 – Presente | Torino, Italia}
\begin{tightemize}
\item Gestione di un team di studenti per la progettazione, implementazione e manutenzione di sistemi IT per biciclette prototipo.
\item Collaborazione con la Divisione Elettronica per lo sviluppo e il test di sistemi embedded e sensori.
\item Supporto alla pianificazione, gestione delle tempistiche e ottimizzazione del budget di progetto.
\item Coordinamento della comunicazione inter-divisione e risoluzione rapida dei problemi tecnici.
\item Sviluppo di strumenti software per il monitoraggio e l’analisi delle prestazioni dei prototipi.
\item Organizzazione di riunioni e workshop per l’integrazione dei nuovi membri e la presentazione dei progressi della divisione.
\end{tightemize}
\sectionsep

\runsubsection{\href{https://www.policumbent.it/en/}{Policumbent}} 
\descript{| Membro Divisione IT} \\
\location{Mar 2025 – Presente | Torino, Italia}
\begin{tightemize}
\item Sviluppo di firmware per i sistemi elettronici a bordo delle biciclette prototipo.
\item Esecuzione di test di sistema per verificare affidabilità e stabilità in condizioni reali.
\item Supporto alla Divisione Elettronica durante le giornate di test in pista e risoluzione di problemi tecnici.
\item Collaborazione alla progettazione e sviluppo di PCB, acquisendo esperienza pratica con l’integrazione hardware.
\end{tightemize}
\sectionsep

%%%%%%%%%%%%%%%%%%%%%%%%%%%%%%%%%%%%%%
% PROGETTI
%%%%%%%%%%%%%%%%%%%%%%%%%%%%%%%%%%%%%%
\section*{Progetti}

\runsubsection{Rowhammer Emulator in QEMU} 
\descript{| Sviluppatore} \\
\location{Mag 2025 – Ott 2025 | Torino, Italia}
\begin{tightemize}
\item Sviluppato un emulatore Rowhammer integrato in QEMU per simulare attacchi in ambienti virtualizzati.
\item Implementati pattern di accesso alla memoria per provocare bit flip nei moduli DRAM.
\item Realizzate varianti di attacco single-sided, double-sided e PARA.
\item Eseguita un’analisi delle prestazioni per valutarne l’efficacia in scenari realistici.
\item Documentata l’architettura dell’emulatore e le linee guida per la ricerca futura nel campo della sicurezza hardware.
\end{tightemize}
\sectionsep

% ----------------------------
% COLONNA DESTRA
% ----------------------------
\switchcolumn

%%%%%%%%%%%%%%%%%%%%%%%%%%%%%%%%%%%%%%
% ISTRUZIONE
%%%%%%%%%%%%%%%%%%%%%%%%%%%%%%%%%%%%%%
\section*{Formazione}

\subsection*{Politecnico di Torino}
\descript{Laurea Magistrale in Ingegneria Informatica — Curriculum Embedded Systems}\\
\location{Prevista: Marzo 2027 | Torino, Italia} \\
DAUIN (Dipartimento di AUtomatica e INformatica)
\sectionsep

\subsection*{Politecnico di Torino}
\descript{Laurea Triennale in Ingegneria Informatica}\\
\location{Luglio 2024 | Torino, Italia} \\
DAUIN (Dipartimento di AUtomatica e INformatica) \\
\location{Voto finale: 93 / 110}
\sectionsep

\subsection*{IIS Leonardo, Giarre}
\descript{Diploma di Liceo Scientifico — Scienze Applicate}\\
\location{Luglio 2020 | Giarre, Catania, Italia} \\
\location{Voto finale: 96 / 100}
\sectionsep

%%%%%%%%%%%%%%%%%%%%%%%%%%%%%%%%%%%%%%
% COMPETENZE
%%%%%%%%%%%%%%%%%%%%%%%%%%%%%%%%%%%%%%
\section*{Competenze}

\subsection*{Linguaggi di Programmazione}
\location{Intermedio (3+ anni):} Python \textbullet{} Bash \textbullet{} C\\
\location{Base (1+ anni):} VHDL \textbullet{} Assembly \textbullet{} LaTeX\\
\location{Conoscenza iniziale:} Verilog
\sectionsep

\subsection*{Tecnologie e Strumenti}
Git / GitHub \textbullet{} Linux / UNIX \textbullet{} Windows \\
Programmazione e debug di microcontrollori \textbullet{} Protocolli seriali (I2C, UART, SPI) \\
CAN Bus \textbullet{} RTOS (FreeRTOS) \textbullet{} Progettazione PCB \textbullet{} Altium Designer
\sectionsep

%%%%%%%%%%%%%%%%%%%%%%%%%%%%%%%%%%%%%%
% LINGUE
%%%%%%%%%%%%%%%%%%%%%%%%%%%%%%%%%%%%%%
\section*{Lingue}
Italiano: Madrelingua \\
Inglese: Avanzato (C2)

\end{paracol}

% ========================
% Pagina 2
% ========================
\clearpage
\begin{paracol}{2}
\setlength{\columnsep}{1cm}
\columnratio{0.6,0.4}

% ----------------------------
% COLONNA SINISTRA
% ----------------------------

\section*{Progetti Aggiuntivi}

\runsubsection{Vertigo e Refuso Firmware} 
\descript{| Sviluppatore} \\
\location{Mar 2025 – Presente | Torino, Italia}
\begin{tightemize}
\item Sviluppo del firmware per Vertigo e Refuso, due PCB utilizzate per acquisizione dati e controllo motore.
\item Implementazione di protocolli di comunicazione (I2C, UART, SPI) per l’interfacciamento dei sensori; Refuso invia i dati del cambio al CAN bus.
\item Sviluppo di algoritmi di controllo motore su Vertigo basati sui dati di Refuso, con cambio sequenziale e invio del rapporto inserito sul CAN bus per analisi e visualizzazione.
\item Test e debug del firmware per garantirne l’affidabilità e le prestazioni in condizioni reali.
\item Collaborazione con ingegneri hardware per ottimizzare l’integrazione firmware-hardware.
\end{tightemize}
\sectionsep

%%%%%%%%%%%%%%%%%%%%%%%%%%%%%%%%%%%%%%
% ALTRE INFORMAZIONI
%%%%%%%%%%%%%%%%%%%%%%%%%%%%%%%%%%%%%%
\section*{Informazioni Aggiuntive}

\subsection*{Collegio Renato Einaudi}
\location{Aprile 2021 – Presente | Torino, Italia} \\
Membro di una residenza universitaria di merito accreditata dal MIUR. Partecipazione a programmi accademici e formativi personalizzati, workshop interdisciplinari, attività culturali e sportive. Sviluppo di competenze trasversali in lavoro di squadra, mentoring e organizzazione di progetti.
\sectionsep

\subsection*{Tutoraggio}
Supporto accademico a studenti più giovani in programmazione, matematica e fisica, assistendo nella preparazione di esami ed esercitazioni. Attività di tutoraggio in Python e C durante gli studi liceali e universitari, promuovendo la comprensione di concetti fondamentali e capacità di problem solving.
\sectionsep

% ----------------------------
% COLONNA DESTRA
% ----------------------------
\switchcolumn

%%%%%%%%%%%%%%%%%%%%%%%%%%%%%%%%%%%%%%
% CORSI UNIVERSITARI
%%%%%%%%%%%%%%%%%%%%%%%%%%%%%%%%%%%%%%
\section*{Corsi Universitari}

\subsection*{Laurea Magistrale}
Computer Architecture II \\
Cybersecurity for Embedded Systems \\
Synthesis and Optimization of Digital Systems \\
Microelectronic Systems \\
Software Engineering \\
Big Data: Architectures and Data Analytics \\
Data Science and Database Technologies
\sectionsep

\subsection*{Laurea Triennale}
Elettromagnetismo e Teoria dei Circuiti \\
Elettronica Applicata \\
Sistemi Elettronici, Tecnologie e Misure \\
Programmazione a Oggetti \\
Controlli Automatici \\
Teoria ed Elaborazione dei Segnali \\
Reti di Calcolatori \\
Architettura dei Calcolatori I \\
Basi di Dati \\
Analisi I-II \\
Algebra Lineare e Geometria \\
Metodi Matematici per l’Ingegneria \\
Sistemi Operativi \\
Algoritmi e Strutture Dati \\
Tecniche di Programmazione \\
Informatica
\sectionsep

\end{paracol}

\end{document}
